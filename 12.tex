\subsection{}
The expectation value is given by:
\begin{equation*}
    \expectationvalue{p_x^2} = \expectationvalue{\widehat{p}_x^2}{\psi_n}
\end{equation*}
Where:
\begin{equation*}
    \widehat{p}_x^2 = - \frac{\hbar^2}{2a^2} \del{\loweringoperator \loweringoperator - \loweringoperator \raisingoperator - \raisingoperator \loweringoperator + \raisingoperator \raisingoperator}
\end{equation*}
So:
\begin{align*}
    \expectationvalue{p_x^2} & = - \frac{\hbar^2}{2a^2} \del{\sqrt{n} \sqrt{n-1} \braket{\psi_n}{\psi_{n-2}} - \del{n+1} \braket{\psi_n} - n \braket{\psi_n} + \sqrt{n+1} \sqrt{n+2} \braket{\psi_n}{\psi_{n+2}}}\\
    & =  - \frac{\hbar^2}{2a^2} \del{- \del{n+1} - n}\\
    & = \frac{\hbar^2}{2a^2} \del{2n + 1}
\end{align*}
For the third excited state, $ n = 3 $:
\begin{equation*}
    \expectationvalue{p_x^2} = \frac{7 \hbar^2}{2a^2}
\end{equation*}
For the fourth excited state, $ n = 4 $:
\begin{equation*}
    \expectationvalue{p_x^2} = \frac{9 \hbar^2}{2a^2}
\end{equation*}

\subsection{}
Given $ E = \del{n + \frac{1}{2}} \hbar \omega_0 $ the expectation of the energy, $ E $, in this state is given by:
\begin{align*}
    \expectationvalue{E} & = \frac{1}{2}E_3 + \frac{1}{2}E_4\\
    & = \frac{7}{4} \hbar \omega_0 + \frac{9}{4} \hbar \omega_0\\
    & = 4 \hbar \omega_0
\end{align*}

\subsection{}
Given $ \widehat{p}_x = -\operatorname{i} \hbar \pd{}{x} $ then $ \widehat{p}_x^2 = - \hbar^2 \pd[2]{}{x} $, which doesn't affect the evenness or oddness of the function on which is operates. So, both integrals have integrands that are the product of an even function and an odd function yielding odd functions. The integral of an odd function over a symmetric range centred on the origin is 0. Hence both integrals are $ = 0 $.

\subsection{}
Given:
\begin{align*}
    \abs{I_{kj}}^2 & = \abs{\int_{-\infty}^{\infty} \psi_k^* \del{x} x \psi_j \del{x} \dif x}^4\\
    \widehat{x} & = \frac{a}{\sqrt{2}} \del{\loweringoperator + \raisingoperator}
\end{align*}
Then:
\begin{align*}
    I_{kj} & = \frac{a^2}{2} \abs{\int_{-\infty}^{\infty} \psi_k^* \del{x} \del{\sqrt{j} \psi_{j-1} \del{x} + \sqrt{j+1} \psi_{j+1} \del{x}} \dif x}^2\\
    & = \frac{a^2}{2} \abs{\sqrt{j} \int_{-\infty}^{\infty} \psi_k^* \del{x} \psi_{j-1} \del{x} \dif x + \sqrt{j+1} \int_{-\infty}^{\infty} \psi_k^* \del{x} \psi_{j+1} \del{x} \dif x}^2
\end{align*}
Which, by orthanormality, is only not equal to zero if $ k = j - 1 $ or $ k = j + 1 $.

\begin{align*}
    \abs{I_{j, j+1}}^2 & = \frac{a^4}{4} \abs{\sqrt{j+1} \int_{-\infty}^{\infty} \psi_j^* \del{x} \psi_j \del{x} \dif x + \sqrt{j+2} \int_{-\infty}^{\infty} \psi_j^* \del{x} \psi_{j+2} \del{x} \dif x}^4\\
    & = \frac{\abs{j + 1}^2 a^4}{4}\\
    \abs{I_{j+1,j}}^2 & = \frac{a^4}{4} \abs{\sqrt{j} \int_{-\infty}^{\infty} \psi_{j+1}^* \del{x} \psi_{j-1} \del{x} \dif x + \sqrt{j+1} \int_{-\infty}^{\infty} \psi_{j+1}^* \del{x} \psi_{j+1} \del{x} \dif x}^4\\
    & = \frac{\abs{j + 1}^2 a^4}{4} = \abs{I_{j, j+1}}^2
\end{align*}
As required.
