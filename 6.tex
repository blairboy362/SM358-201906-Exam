\subsection{}
If an atom has an even number of constituent fermions, it behaves as a fermion; if it has an odd number of constituent fermions, it behaves as a boson. \ce{^{40}K} has an odd number of constituent fermions (59) and so behaves as a fermion.

\subsection{}
Fermions are described by an anti-symmetric total wave function. The spin ket given is anti-symmetric. Therefore the spatial wave function must be symmetric. The electrons will tend to crowd together.

\subsection{}
\ce{^{41}K} acts as a boson (it has an even number of constituent fermions). Bosons are described by a symmetric total wave function. If the two particles of \ce{^{41}K} are in a symmetric spin state the spatial wave function must also be symmetric, so the answer to b) would be the same; the electrons will tend to crowd together.
